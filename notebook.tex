
% Default to the notebook output style

    


% Inherit from the specified cell style.




    
\documentclass[11pt]{article}

    
    
    \usepackage[T1]{fontenc}
    % Nicer default font (+ math font) than Computer Modern for most use cases
    \usepackage{mathpazo}

    % Basic figure setup, for now with no caption control since it's done
    % automatically by Pandoc (which extracts ![](path) syntax from Markdown).
    \usepackage{graphicx}
    % We will generate all images so they have a width \maxwidth. This means
    % that they will get their normal width if they fit onto the page, but
    % are scaled down if they would overflow the margins.
    \makeatletter
    \def\maxwidth{\ifdim\Gin@nat@width>\linewidth\linewidth
    \else\Gin@nat@width\fi}
    \makeatother
    \let\Oldincludegraphics\includegraphics
    % Set max figure width to be 80% of text width, for now hardcoded.
    \renewcommand{\includegraphics}[1]{\Oldincludegraphics[width=.8\maxwidth]{#1}}
    % Ensure that by default, figures have no caption (until we provide a
    % proper Figure object with a Caption API and a way to capture that
    % in the conversion process - todo).
    \usepackage{caption}
    \DeclareCaptionLabelFormat{nolabel}{}
    \captionsetup{labelformat=nolabel}

    \usepackage{adjustbox} % Used to constrain images to a maximum size 
    \usepackage{xcolor} % Allow colors to be defined
    \usepackage{enumerate} % Needed for markdown enumerations to work
    \usepackage{geometry} % Used to adjust the document margins
    \usepackage{amsmath} % Equations
    \usepackage{amssymb} % Equations
    \usepackage{textcomp} % defines textquotesingle
    % Hack from http://tex.stackexchange.com/a/47451/13684:
    \AtBeginDocument{%
        \def\PYZsq{\textquotesingle}% Upright quotes in Pygmentized code
    }
    \usepackage{upquote} % Upright quotes for verbatim code
    \usepackage{eurosym} % defines \euro
    \usepackage[mathletters]{ucs} % Extended unicode (utf-8) support
    \usepackage[utf8x]{inputenc} % Allow utf-8 characters in the tex document
    \usepackage{fancyvrb} % verbatim replacement that allows latex
    \usepackage{grffile} % extends the file name processing of package graphics 
                         % to support a larger range 
    % The hyperref package gives us a pdf with properly built
    % internal navigation ('pdf bookmarks' for the table of contents,
    % internal cross-reference links, web links for URLs, etc.)
    \usepackage{hyperref}
    \usepackage{longtable} % longtable support required by pandoc >1.10
    \usepackage{booktabs}  % table support for pandoc > 1.12.2
    \usepackage[inline]{enumitem} % IRkernel/repr support (it uses the enumerate* environment)
    \usepackage[normalem]{ulem} % ulem is needed to support strikethroughs (\sout)
                                % normalem makes italics be italics, not underlines
    

    
    
    % Colors for the hyperref package
    \definecolor{urlcolor}{rgb}{0,.145,.698}
    \definecolor{linkcolor}{rgb}{.71,0.21,0.01}
    \definecolor{citecolor}{rgb}{.12,.54,.11}

    % ANSI colors
    \definecolor{ansi-black}{HTML}{3E424D}
    \definecolor{ansi-black-intense}{HTML}{282C36}
    \definecolor{ansi-red}{HTML}{E75C58}
    \definecolor{ansi-red-intense}{HTML}{B22B31}
    \definecolor{ansi-green}{HTML}{00A250}
    \definecolor{ansi-green-intense}{HTML}{007427}
    \definecolor{ansi-yellow}{HTML}{DDB62B}
    \definecolor{ansi-yellow-intense}{HTML}{B27D12}
    \definecolor{ansi-blue}{HTML}{208FFB}
    \definecolor{ansi-blue-intense}{HTML}{0065CA}
    \definecolor{ansi-magenta}{HTML}{D160C4}
    \definecolor{ansi-magenta-intense}{HTML}{A03196}
    \definecolor{ansi-cyan}{HTML}{60C6C8}
    \definecolor{ansi-cyan-intense}{HTML}{258F8F}
    \definecolor{ansi-white}{HTML}{C5C1B4}
    \definecolor{ansi-white-intense}{HTML}{A1A6B2}

    % commands and environments needed by pandoc snippets
    % extracted from the output of `pandoc -s`
    \providecommand{\tightlist}{%
      \setlength{\itemsep}{0pt}\setlength{\parskip}{0pt}}
    \DefineVerbatimEnvironment{Highlighting}{Verbatim}{commandchars=\\\{\}}
    % Add ',fontsize=\small' for more characters per line
    \newenvironment{Shaded}{}{}
    \newcommand{\KeywordTok}[1]{\textcolor[rgb]{0.00,0.44,0.13}{\textbf{{#1}}}}
    \newcommand{\DataTypeTok}[1]{\textcolor[rgb]{0.56,0.13,0.00}{{#1}}}
    \newcommand{\DecValTok}[1]{\textcolor[rgb]{0.25,0.63,0.44}{{#1}}}
    \newcommand{\BaseNTok}[1]{\textcolor[rgb]{0.25,0.63,0.44}{{#1}}}
    \newcommand{\FloatTok}[1]{\textcolor[rgb]{0.25,0.63,0.44}{{#1}}}
    \newcommand{\CharTok}[1]{\textcolor[rgb]{0.25,0.44,0.63}{{#1}}}
    \newcommand{\StringTok}[1]{\textcolor[rgb]{0.25,0.44,0.63}{{#1}}}
    \newcommand{\CommentTok}[1]{\textcolor[rgb]{0.38,0.63,0.69}{\textit{{#1}}}}
    \newcommand{\OtherTok}[1]{\textcolor[rgb]{0.00,0.44,0.13}{{#1}}}
    \newcommand{\AlertTok}[1]{\textcolor[rgb]{1.00,0.00,0.00}{\textbf{{#1}}}}
    \newcommand{\FunctionTok}[1]{\textcolor[rgb]{0.02,0.16,0.49}{{#1}}}
    \newcommand{\RegionMarkerTok}[1]{{#1}}
    \newcommand{\ErrorTok}[1]{\textcolor[rgb]{1.00,0.00,0.00}{\textbf{{#1}}}}
    \newcommand{\NormalTok}[1]{{#1}}
    
    % Additional commands for more recent versions of Pandoc
    \newcommand{\ConstantTok}[1]{\textcolor[rgb]{0.53,0.00,0.00}{{#1}}}
    \newcommand{\SpecialCharTok}[1]{\textcolor[rgb]{0.25,0.44,0.63}{{#1}}}
    \newcommand{\VerbatimStringTok}[1]{\textcolor[rgb]{0.25,0.44,0.63}{{#1}}}
    \newcommand{\SpecialStringTok}[1]{\textcolor[rgb]{0.73,0.40,0.53}{{#1}}}
    \newcommand{\ImportTok}[1]{{#1}}
    \newcommand{\DocumentationTok}[1]{\textcolor[rgb]{0.73,0.13,0.13}{\textit{{#1}}}}
    \newcommand{\AnnotationTok}[1]{\textcolor[rgb]{0.38,0.63,0.69}{\textbf{\textit{{#1}}}}}
    \newcommand{\CommentVarTok}[1]{\textcolor[rgb]{0.38,0.63,0.69}{\textbf{\textit{{#1}}}}}
    \newcommand{\VariableTok}[1]{\textcolor[rgb]{0.10,0.09,0.49}{{#1}}}
    \newcommand{\ControlFlowTok}[1]{\textcolor[rgb]{0.00,0.44,0.13}{\textbf{{#1}}}}
    \newcommand{\OperatorTok}[1]{\textcolor[rgb]{0.40,0.40,0.40}{{#1}}}
    \newcommand{\BuiltInTok}[1]{{#1}}
    \newcommand{\ExtensionTok}[1]{{#1}}
    \newcommand{\PreprocessorTok}[1]{\textcolor[rgb]{0.74,0.48,0.00}{{#1}}}
    \newcommand{\AttributeTok}[1]{\textcolor[rgb]{0.49,0.56,0.16}{{#1}}}
    \newcommand{\InformationTok}[1]{\textcolor[rgb]{0.38,0.63,0.69}{\textbf{\textit{{#1}}}}}
    \newcommand{\WarningTok}[1]{\textcolor[rgb]{0.38,0.63,0.69}{\textbf{\textit{{#1}}}}}
    
    
    % Define a nice break command that doesn't care if a line doesn't already
    % exist.
    \def\br{\hspace*{\fill} \\* }
    % Math Jax compatability definitions
    \def\gt{>}
    \def\lt{<}
    % Document parameters
    \title{Part 1 - Introduction to Python and Jupyter}
    
    
    

    % Pygments definitions
    
\makeatletter
\def\PY@reset{\let\PY@it=\relax \let\PY@bf=\relax%
    \let\PY@ul=\relax \let\PY@tc=\relax%
    \let\PY@bc=\relax \let\PY@ff=\relax}
\def\PY@tok#1{\csname PY@tok@#1\endcsname}
\def\PY@toks#1+{\ifx\relax#1\empty\else%
    \PY@tok{#1}\expandafter\PY@toks\fi}
\def\PY@do#1{\PY@bc{\PY@tc{\PY@ul{%
    \PY@it{\PY@bf{\PY@ff{#1}}}}}}}
\def\PY#1#2{\PY@reset\PY@toks#1+\relax+\PY@do{#2}}

\expandafter\def\csname PY@tok@gd\endcsname{\def\PY@tc##1{\textcolor[rgb]{0.63,0.00,0.00}{##1}}}
\expandafter\def\csname PY@tok@gu\endcsname{\let\PY@bf=\textbf\def\PY@tc##1{\textcolor[rgb]{0.50,0.00,0.50}{##1}}}
\expandafter\def\csname PY@tok@gt\endcsname{\def\PY@tc##1{\textcolor[rgb]{0.00,0.27,0.87}{##1}}}
\expandafter\def\csname PY@tok@gs\endcsname{\let\PY@bf=\textbf}
\expandafter\def\csname PY@tok@gr\endcsname{\def\PY@tc##1{\textcolor[rgb]{1.00,0.00,0.00}{##1}}}
\expandafter\def\csname PY@tok@cm\endcsname{\let\PY@it=\textit\def\PY@tc##1{\textcolor[rgb]{0.25,0.50,0.50}{##1}}}
\expandafter\def\csname PY@tok@vg\endcsname{\def\PY@tc##1{\textcolor[rgb]{0.10,0.09,0.49}{##1}}}
\expandafter\def\csname PY@tok@vi\endcsname{\def\PY@tc##1{\textcolor[rgb]{0.10,0.09,0.49}{##1}}}
\expandafter\def\csname PY@tok@vm\endcsname{\def\PY@tc##1{\textcolor[rgb]{0.10,0.09,0.49}{##1}}}
\expandafter\def\csname PY@tok@mh\endcsname{\def\PY@tc##1{\textcolor[rgb]{0.40,0.40,0.40}{##1}}}
\expandafter\def\csname PY@tok@cs\endcsname{\let\PY@it=\textit\def\PY@tc##1{\textcolor[rgb]{0.25,0.50,0.50}{##1}}}
\expandafter\def\csname PY@tok@ge\endcsname{\let\PY@it=\textit}
\expandafter\def\csname PY@tok@vc\endcsname{\def\PY@tc##1{\textcolor[rgb]{0.10,0.09,0.49}{##1}}}
\expandafter\def\csname PY@tok@il\endcsname{\def\PY@tc##1{\textcolor[rgb]{0.40,0.40,0.40}{##1}}}
\expandafter\def\csname PY@tok@go\endcsname{\def\PY@tc##1{\textcolor[rgb]{0.53,0.53,0.53}{##1}}}
\expandafter\def\csname PY@tok@cp\endcsname{\def\PY@tc##1{\textcolor[rgb]{0.74,0.48,0.00}{##1}}}
\expandafter\def\csname PY@tok@gi\endcsname{\def\PY@tc##1{\textcolor[rgb]{0.00,0.63,0.00}{##1}}}
\expandafter\def\csname PY@tok@gh\endcsname{\let\PY@bf=\textbf\def\PY@tc##1{\textcolor[rgb]{0.00,0.00,0.50}{##1}}}
\expandafter\def\csname PY@tok@ni\endcsname{\let\PY@bf=\textbf\def\PY@tc##1{\textcolor[rgb]{0.60,0.60,0.60}{##1}}}
\expandafter\def\csname PY@tok@nl\endcsname{\def\PY@tc##1{\textcolor[rgb]{0.63,0.63,0.00}{##1}}}
\expandafter\def\csname PY@tok@nn\endcsname{\let\PY@bf=\textbf\def\PY@tc##1{\textcolor[rgb]{0.00,0.00,1.00}{##1}}}
\expandafter\def\csname PY@tok@no\endcsname{\def\PY@tc##1{\textcolor[rgb]{0.53,0.00,0.00}{##1}}}
\expandafter\def\csname PY@tok@na\endcsname{\def\PY@tc##1{\textcolor[rgb]{0.49,0.56,0.16}{##1}}}
\expandafter\def\csname PY@tok@nb\endcsname{\def\PY@tc##1{\textcolor[rgb]{0.00,0.50,0.00}{##1}}}
\expandafter\def\csname PY@tok@nc\endcsname{\let\PY@bf=\textbf\def\PY@tc##1{\textcolor[rgb]{0.00,0.00,1.00}{##1}}}
\expandafter\def\csname PY@tok@nd\endcsname{\def\PY@tc##1{\textcolor[rgb]{0.67,0.13,1.00}{##1}}}
\expandafter\def\csname PY@tok@ne\endcsname{\let\PY@bf=\textbf\def\PY@tc##1{\textcolor[rgb]{0.82,0.25,0.23}{##1}}}
\expandafter\def\csname PY@tok@nf\endcsname{\def\PY@tc##1{\textcolor[rgb]{0.00,0.00,1.00}{##1}}}
\expandafter\def\csname PY@tok@si\endcsname{\let\PY@bf=\textbf\def\PY@tc##1{\textcolor[rgb]{0.73,0.40,0.53}{##1}}}
\expandafter\def\csname PY@tok@s2\endcsname{\def\PY@tc##1{\textcolor[rgb]{0.73,0.13,0.13}{##1}}}
\expandafter\def\csname PY@tok@nt\endcsname{\let\PY@bf=\textbf\def\PY@tc##1{\textcolor[rgb]{0.00,0.50,0.00}{##1}}}
\expandafter\def\csname PY@tok@nv\endcsname{\def\PY@tc##1{\textcolor[rgb]{0.10,0.09,0.49}{##1}}}
\expandafter\def\csname PY@tok@s1\endcsname{\def\PY@tc##1{\textcolor[rgb]{0.73,0.13,0.13}{##1}}}
\expandafter\def\csname PY@tok@dl\endcsname{\def\PY@tc##1{\textcolor[rgb]{0.73,0.13,0.13}{##1}}}
\expandafter\def\csname PY@tok@ch\endcsname{\let\PY@it=\textit\def\PY@tc##1{\textcolor[rgb]{0.25,0.50,0.50}{##1}}}
\expandafter\def\csname PY@tok@m\endcsname{\def\PY@tc##1{\textcolor[rgb]{0.40,0.40,0.40}{##1}}}
\expandafter\def\csname PY@tok@gp\endcsname{\let\PY@bf=\textbf\def\PY@tc##1{\textcolor[rgb]{0.00,0.00,0.50}{##1}}}
\expandafter\def\csname PY@tok@sh\endcsname{\def\PY@tc##1{\textcolor[rgb]{0.73,0.13,0.13}{##1}}}
\expandafter\def\csname PY@tok@ow\endcsname{\let\PY@bf=\textbf\def\PY@tc##1{\textcolor[rgb]{0.67,0.13,1.00}{##1}}}
\expandafter\def\csname PY@tok@sx\endcsname{\def\PY@tc##1{\textcolor[rgb]{0.00,0.50,0.00}{##1}}}
\expandafter\def\csname PY@tok@bp\endcsname{\def\PY@tc##1{\textcolor[rgb]{0.00,0.50,0.00}{##1}}}
\expandafter\def\csname PY@tok@c1\endcsname{\let\PY@it=\textit\def\PY@tc##1{\textcolor[rgb]{0.25,0.50,0.50}{##1}}}
\expandafter\def\csname PY@tok@fm\endcsname{\def\PY@tc##1{\textcolor[rgb]{0.00,0.00,1.00}{##1}}}
\expandafter\def\csname PY@tok@o\endcsname{\def\PY@tc##1{\textcolor[rgb]{0.40,0.40,0.40}{##1}}}
\expandafter\def\csname PY@tok@kc\endcsname{\let\PY@bf=\textbf\def\PY@tc##1{\textcolor[rgb]{0.00,0.50,0.00}{##1}}}
\expandafter\def\csname PY@tok@c\endcsname{\let\PY@it=\textit\def\PY@tc##1{\textcolor[rgb]{0.25,0.50,0.50}{##1}}}
\expandafter\def\csname PY@tok@mf\endcsname{\def\PY@tc##1{\textcolor[rgb]{0.40,0.40,0.40}{##1}}}
\expandafter\def\csname PY@tok@err\endcsname{\def\PY@bc##1{\setlength{\fboxsep}{0pt}\fcolorbox[rgb]{1.00,0.00,0.00}{1,1,1}{\strut ##1}}}
\expandafter\def\csname PY@tok@mb\endcsname{\def\PY@tc##1{\textcolor[rgb]{0.40,0.40,0.40}{##1}}}
\expandafter\def\csname PY@tok@ss\endcsname{\def\PY@tc##1{\textcolor[rgb]{0.10,0.09,0.49}{##1}}}
\expandafter\def\csname PY@tok@sr\endcsname{\def\PY@tc##1{\textcolor[rgb]{0.73,0.40,0.53}{##1}}}
\expandafter\def\csname PY@tok@mo\endcsname{\def\PY@tc##1{\textcolor[rgb]{0.40,0.40,0.40}{##1}}}
\expandafter\def\csname PY@tok@kd\endcsname{\let\PY@bf=\textbf\def\PY@tc##1{\textcolor[rgb]{0.00,0.50,0.00}{##1}}}
\expandafter\def\csname PY@tok@mi\endcsname{\def\PY@tc##1{\textcolor[rgb]{0.40,0.40,0.40}{##1}}}
\expandafter\def\csname PY@tok@kn\endcsname{\let\PY@bf=\textbf\def\PY@tc##1{\textcolor[rgb]{0.00,0.50,0.00}{##1}}}
\expandafter\def\csname PY@tok@cpf\endcsname{\let\PY@it=\textit\def\PY@tc##1{\textcolor[rgb]{0.25,0.50,0.50}{##1}}}
\expandafter\def\csname PY@tok@kr\endcsname{\let\PY@bf=\textbf\def\PY@tc##1{\textcolor[rgb]{0.00,0.50,0.00}{##1}}}
\expandafter\def\csname PY@tok@s\endcsname{\def\PY@tc##1{\textcolor[rgb]{0.73,0.13,0.13}{##1}}}
\expandafter\def\csname PY@tok@kp\endcsname{\def\PY@tc##1{\textcolor[rgb]{0.00,0.50,0.00}{##1}}}
\expandafter\def\csname PY@tok@w\endcsname{\def\PY@tc##1{\textcolor[rgb]{0.73,0.73,0.73}{##1}}}
\expandafter\def\csname PY@tok@kt\endcsname{\def\PY@tc##1{\textcolor[rgb]{0.69,0.00,0.25}{##1}}}
\expandafter\def\csname PY@tok@sc\endcsname{\def\PY@tc##1{\textcolor[rgb]{0.73,0.13,0.13}{##1}}}
\expandafter\def\csname PY@tok@sb\endcsname{\def\PY@tc##1{\textcolor[rgb]{0.73,0.13,0.13}{##1}}}
\expandafter\def\csname PY@tok@sa\endcsname{\def\PY@tc##1{\textcolor[rgb]{0.73,0.13,0.13}{##1}}}
\expandafter\def\csname PY@tok@k\endcsname{\let\PY@bf=\textbf\def\PY@tc##1{\textcolor[rgb]{0.00,0.50,0.00}{##1}}}
\expandafter\def\csname PY@tok@se\endcsname{\let\PY@bf=\textbf\def\PY@tc##1{\textcolor[rgb]{0.73,0.40,0.13}{##1}}}
\expandafter\def\csname PY@tok@sd\endcsname{\let\PY@it=\textit\def\PY@tc##1{\textcolor[rgb]{0.73,0.13,0.13}{##1}}}

\def\PYZbs{\char`\\}
\def\PYZus{\char`\_}
\def\PYZob{\char`\{}
\def\PYZcb{\char`\}}
\def\PYZca{\char`\^}
\def\PYZam{\char`\&}
\def\PYZlt{\char`\<}
\def\PYZgt{\char`\>}
\def\PYZsh{\char`\#}
\def\PYZpc{\char`\%}
\def\PYZdl{\char`\$}
\def\PYZhy{\char`\-}
\def\PYZsq{\char`\'}
\def\PYZdq{\char`\"}
\def\PYZti{\char`\~}
% for compatibility with earlier versions
\def\PYZat{@}
\def\PYZlb{[}
\def\PYZrb{]}
\makeatother


    % Exact colors from NB
    \definecolor{incolor}{rgb}{0.0, 0.0, 0.5}
    \definecolor{outcolor}{rgb}{0.545, 0.0, 0.0}



    
    % Prevent overflowing lines due to hard-to-break entities
    \sloppy 
    % Setup hyperref package
    \hypersetup{
      breaklinks=true,  % so long urls are correctly broken across lines
      colorlinks=true,
      urlcolor=urlcolor,
      linkcolor=linkcolor,
      citecolor=citecolor,
      }
    % Slightly bigger margins than the latex defaults
    
    \geometry{verbose,tmargin=1in,bmargin=1in,lmargin=1in,rmargin=1in}
    
    

    \begin{document}
    
    
    \maketitle
    
    

    
    Table of Contents{}

{{1~~}Why this topic, why now?}

{{2~~}Introduction to Jupyter}

{{2.1~~}Overview of Jupyter}

{{2.2~~}Jupyter Kernels (Language Support)}

{{2.3~~}Getting up and running in Jupyter}

{{2.4~~}You can also}

{{2.5~~}Markdown}

{{2.5.1~~}Markdown syntax examples}

{{2.6~~}Jupyter Pros and Cons}

{{2.7~~}JupyterLab (In Beta)}

{{3~~}Overview of Python}

{{3.1~~}Python 2 vs 3}

{{3.2~~}Numeric/Scientific Computing in Python}

{{3.2.1~~}Example Numpy modules}

{{3.2.2~~}Example Scipy Modules}

{{3.2.3~~}Numpy for Matlab Users}

{{4~~}Python Syntax}

{{4.1~~}Variables}

{{4.1.1~~}Define and Print Variables}

{{4.1.2~~}Boolean}

{{4.1.3~~}Strings}

{{4.1.4~~}Lists}

{{4.1.5~~}Dictionaries}

{{4.2~~}Math Operators}

{{4.2.1~~}Working with integer and float variables}

{{4.3~~}Variable Values and References}

{{4.4~~}Loops}

{{4.4.1~~}Standard for Loop}

{{4.4.2~~}for Loop using enumerate}

{{4.4.3~~}for Loop using Dictionaries}

{{4.4.4~~}List comprehension}

{{4.5~~}Conditionals}

{{4.6~~}Functions}

{{4.6.1~~}Standard Functions}

{{4.6.2~~}Functions that return values}

{{4.6.3~~}Shorthand Functions (lambda)}

{{4.7~~}Imports}

{{4.8~~}Helpful configurations}

{{4.9~~}Intro to Numpy}

{{4.9.1~~}Numpy Math}

{{4.9.2~~}Numpy Arrays and Matrices}

{{4.10~~}Random Numbers and Plotting}

{{4.11~~}Pandas}

    \section{Why this topic, why now?}\label{why-this-topic-why-now}

\begin{itemize}
\tightlist
\item
  Jupyter allows for living computational documents
\item
  Explosion in use and new libraries in Python
\item
  Streamlined sharing of knowledge
\item
  Real-time interactive learning
\item
  Compute Canada launched a free Jupyter instance about a year ago
\end{itemize}

    \section{Introduction to Jupyter}\label{introduction-to-jupyter}

    https://github.com/cisl/python-jupyter-intro

    \subsection{Overview of Jupyter}\label{overview-of-jupyter}

\begin{itemize}
\tightlist
\item
  Jupyter is a web-based tool for writing code, documentation,
  tutorials, etc.
\item
  What you are reading was created in
  \href{http://jupyter.org/}{Jupyter}
\item
  You can access a free Compute-Canada hosted version of Jupyter at
  https://uvic.syzygy.ca/.
\end{itemize}

    \subsection{Jupyter Kernels (Language
Support)}\label{jupyter-kernels-language-support}

\begin{itemize}
\tightlist
\item
  Markdown
\item
  Python
\item
  Matlab
\item
  R
\item
  Fortran
\item
  C/C++
\item
  Julia
\item
  Latex Expressions (MathJax)
\item
  \textasciitilde{}70 in total
  (https://github.com/jupyter/jupyter/wiki/Jupyter-kernels)
\end{itemize}

    \subsection{Getting up and running in
Jupyter}\label{getting-up-and-running-in-jupyter}

\begin{itemize}
\tightlist
\item
  Login to Syzygy with your uvic id.

  \begin{itemize}
  \tightlist
  \item
    \textbf{Click "Keep me signed in for 8 hours" or your session will
    fail shortly after login}
  \item
    \href{mailto:jupyter@pims.math.ca}{\nolinkurl{jupyter@pims.math.ca}}
    for support
  \end{itemize}
\item
  Create (or upload) a notebook.
\item
  Write and run your code
\end{itemize}

    \subsection{You can also}\label{you-can-also}

\begin{itemize}
\tightlist
\item
  Install Python on your computer and use it directly without Jupyter.
\item
  Install Python and Jupyter on your computer and use Jupyter locally
\item
  Use other languages with Jupyter...
\end{itemize}

    \subsection{Markdown}\label{markdown}

\begin{itemize}
\tightlist
\item
  Used to write text that can be easily formatted into HTML
\item
  In Jupyter you can intermix markdown and code cells to create a living
  and executable document
\item
  A Jupyter notebook can be exported as HTML, Latex, PDF (via Latex),
  code, used as slides
\end{itemize}

    \subsubsection{Markdown syntax examples}\label{markdown-syntax-examples}

\begin{verbatim}
# Heading 1
## Heading 2

* Bullet item
    * Sub-item

**bold** *italics*
[Link](www.jupyter.org)
![Alt text](image.jpg)

a|b
---|---
c|d
\end{verbatim}

Latex Expressions: \texttt{\$e\^{}\{i\textbackslash{}pi\}=0\$},
\(e^{i\pi}=0\)

    \subsection{Jupyter Pros and Cons}\label{jupyter-pros-and-cons}

\begin{itemize}
\tightlist
\item
  Pros

  \begin{itemize}
  \tightlist
  \item
    Integrated documenting, coding and executing
  \item
    Support for programming languages and latex
  \item
    A lot of plugins (e.g., for section numbering, Zotero)
  \end{itemize}
\item
  Cons

  \begin{itemize}
  \tightlist
  \item
    Slide functionality is problematic - mostly text-over-run
  \end{itemize}
\end{itemize}

    \subsection{JupyterLab (In Beta)}\label{jupyterlab-in-beta}

Towards an integrated development environment for Jupyter

    \section{Overview of Python}\label{overview-of-python}

\begin{itemize}
\tightlist
\item
  Used widely in academia and industry for data science and numerical
  analysis (among other uses)
\item
  Designed to be simpler than C/Java
\item
  There are \textasciitilde{}2,506,000 open-source python repositories
  on GitHub
\item
  There are \textasciitilde{}134,000 packages across all versions in the
  Python Package Index (i.e., packages that can be installed in one
  command, e.g., \emph{pip install numpy})
\end{itemize}

    \subsection{Python 2 vs 3}\label{python-2-vs-3}

\begin{itemize}
\tightlist
\item
  This tutorial uses Python 2 because I have a lot of code in Python 2
  and only run it - but I should switch
\item
  If you are just getting started, use Python 3.
\end{itemize}

    \subsection{Numeric/Scientific Computing in
Python}\label{numericscientific-computing-in-python}

\begin{itemize}
\tightlist
\item
  \textbf{Numpy} and \textbf{Scipy} provide the fundamentals
\item
  There are other, more specilaiized libraries for machine learning,
  control systems, signal processing, networks/graphs etc.
\end{itemize}

    \subsubsection{Example Numpy modules}\label{example-numpy-modules}

\begin{itemize}
\tightlist
\item
  Array/Matrix objects
\item
  Sorting, searching, and counting
\item
  Linear algebra
\item
  Discrete Fourier Transform
\item
  Polynomials
\item
  Random sampling
\item
  Logic functions
\item
  Financial functions
\end{itemize}

    \subsubsection{Example Scipy Modules}\label{example-scipy-modules}

\begin{itemize}
\tightlist
\item
  Integration (scipy.integrate)
\item
  Optimization (scipy.optimize)
\item
  Interpolation (scipy.interpolate)
\item
  Fourier Transforms (scipy.fftpack)
\item
  Signal Processing (scipy.signal)
\item
  Linear Algebra (scipy.linalg)
\end{itemize}

    \subsubsection{Numpy for Matlab Users}\label{numpy-for-matlab-users}

https://docs.scipy.org/doc/numpy-dev/user/numpy-for-matlab-users.html

    \section{Python Syntax}\label{python-syntax}

If you cannot figure out how to do something in Python just google it.
Most questions have been asked and answered

    \subsection{Variables}\label{variables}

    \subsubsection{Define and Print
Variables}\label{define-and-print-variables}

    \begin{Verbatim}[commandchars=\\\{\}]
{\color{incolor}In [{\color{incolor}52}]:} \PY{n}{x} \PY{o}{=} \PY{l+m+mi}{4} \PY{c+c1}{\PYZsh{} integer}
         \PY{n}{y} \PY{o}{=} \PY{l+m+mf}{3.34523534532123421} \PY{c+c1}{\PYZsh{} float}
\end{Verbatim}


    \begin{Verbatim}[commandchars=\\\{\}]
{\color{incolor}In [{\color{incolor}53}]:} \PY{k}{print} \PY{n}{x}
         \PY{k}{print} \PY{n}{y}
         \PY{k}{print} \PY{l+s+s1}{\PYZsq{}}\PY{l+s+s1}{\PYZob{}:.3g\PYZcb{}}\PY{l+s+s1}{\PYZsq{}}\PY{o}{.}\PY{n}{format}\PY{p}{(}\PY{n}{y}\PY{p}{)}
\end{Verbatim}


    \begin{Verbatim}[commandchars=\\\{\}]
4
3.34523534532
3.35

    \end{Verbatim}

    \begin{Verbatim}[commandchars=\\\{\}]
{\color{incolor}In [{\color{incolor}54}]:} \PY{k}{print} \PY{l+s+s1}{\PYZsq{}}\PY{l+s+s1}{\PYZob{}:.3g\PYZcb{} \PYZob{}:.3g\PYZcb{}}\PY{l+s+s1}{\PYZsq{}}\PY{o}{.}\PY{n}{format}\PY{p}{(}\PY{n}{x}\PY{p}{,}\PY{n}{y}\PY{p}{)}
\end{Verbatim}


    \begin{Verbatim}[commandchars=\\\{\}]
4 3.35

    \end{Verbatim}

    \subsubsection{Boolean}\label{boolean}

    \begin{Verbatim}[commandchars=\\\{\}]
{\color{incolor}In [{\color{incolor}55}]:} \PY{n}{x} \PY{o}{=} \PY{n+nb+bp}{True}
         \PY{n}{y} \PY{o}{=} \PY{n+nb+bp}{False}
         
         \PY{k}{print} \PY{n}{x}
         \PY{k}{print} \PY{n}{y}
         \PY{k}{print} \PY{n}{x}\PY{o}{==}\PY{n}{y}
\end{Verbatim}


    \begin{Verbatim}[commandchars=\\\{\}]
True
False
False

    \end{Verbatim}

    \subsubsection{Strings}\label{strings}

    \begin{Verbatim}[commandchars=\\\{\}]
{\color{incolor}In [{\color{incolor}56}]:} \PY{n}{x} \PY{o}{=} \PY{l+s+s1}{\PYZsq{}}\PY{l+s+s1}{Hello World}\PY{l+s+s1}{\PYZsq{}}
         \PY{n}{y} \PY{o}{=} \PY{l+s+s1}{\PYZsq{}}\PY{l+s+s1}{Hello \PYZob{}\PYZcb{} World \PYZob{}\PYZcb{}}\PY{l+s+s1}{\PYZsq{}}\PY{o}{.}\PY{n}{format}\PY{p}{(}\PY{l+s+s1}{\PYZsq{}}\PY{l+s+s1}{Big}\PY{l+s+s1}{\PYZsq{}}\PY{p}{,}\PY{l+m+mi}{2}\PY{p}{)}
         
         \PY{k}{print} \PY{n}{x}
         \PY{k}{print} \PY{n}{y}
         \PY{k}{print} \PY{n}{x}\PY{o}{+}\PY{n}{y}
\end{Verbatim}


    \begin{Verbatim}[commandchars=\\\{\}]
Hello World
Hello Big World 2
Hello WorldHello Big World 2

    \end{Verbatim}

    \subsubsection{Lists}\label{lists}

    \begin{Verbatim}[commandchars=\\\{\}]
{\color{incolor}In [{\color{incolor}57}]:} \PY{n}{l} \PY{o}{=} \PY{p}{[}\PY{l+m+mi}{1}\PY{p}{,}\PY{l+m+mi}{2}\PY{p}{,}\PY{l+m+mi}{3}\PY{p}{,}\PY{l+m+mi}{4}\PY{p}{]}
\end{Verbatim}


    \begin{Verbatim}[commandchars=\\\{\}]
{\color{incolor}In [{\color{incolor}58}]:} \PY{k}{print} \PY{n}{l}\PY{p}{[}\PY{l+m+mi}{0}\PY{p}{]} \PY{c+c1}{\PYZsh{} access item at index 0}
\end{Verbatim}


    \begin{Verbatim}[commandchars=\\\{\}]
1

    \end{Verbatim}

    \begin{Verbatim}[commandchars=\\\{\}]
{\color{incolor}In [{\color{incolor}59}]:} \PY{k}{print} \PY{n}{l}\PY{o}{+}\PY{n}{l}
\end{Verbatim}


    \begin{Verbatim}[commandchars=\\\{\}]
[1, 2, 3, 4, 1, 2, 3, 4]

    \end{Verbatim}

    \begin{Verbatim}[commandchars=\\\{\}]
{\color{incolor}In [{\color{incolor}60}]:} \PY{k}{print} \PY{n+nb}{len}\PY{p}{(}\PY{n}{l}\PY{p}{)} \PY{c+c1}{\PYZsh{} get the length of the list}
\end{Verbatim}


    \begin{Verbatim}[commandchars=\\\{\}]
4

    \end{Verbatim}

    \begin{Verbatim}[commandchars=\\\{\}]
{\color{incolor}In [{\color{incolor}61}]:} \PY{k}{print} \PY{n}{l}\PY{p}{[}\PY{p}{:}\PY{p}{:}\PY{o}{\PYZhy{}}\PY{l+m+mi}{1}\PY{p}{]} \PY{c+c1}{\PYZsh{} get the list in reverse order}
\end{Verbatim}


    \begin{Verbatim}[commandchars=\\\{\}]
[4, 3, 2, 1]

    \end{Verbatim}

    \begin{Verbatim}[commandchars=\\\{\}]
{\color{incolor}In [{\color{incolor}62}]:} \PY{k}{print} \PY{n+nb}{max}\PY{p}{(}\PY{n}{l}\PY{p}{)}\PY{p}{,} \PY{n+nb}{min}\PY{p}{(}\PY{n}{l}\PY{p}{)}\PY{p}{,} \PY{n+nb}{sum}\PY{p}{(}\PY{n}{l}\PY{p}{)}
\end{Verbatim}


    \begin{Verbatim}[commandchars=\\\{\}]
4 1 10

    \end{Verbatim}

    \begin{Verbatim}[commandchars=\\\{\}]
{\color{incolor}In [{\color{incolor}63}]:} \PY{n}{y} \PY{o}{=} \PY{p}{[}\PY{l+m+mi}{4}\PY{p}{,}\PY{l+m+mi}{2}\PY{p}{,}\PY{l+m+mi}{3}\PY{p}{,}\PY{l+m+mi}{1}\PY{p}{]}
         \PY{n}{y}\PY{o}{.}\PY{n}{sort}\PY{p}{(}\PY{p}{)}
         \PY{k}{print} \PY{n}{y}
\end{Verbatim}


    \begin{Verbatim}[commandchars=\\\{\}]
[1, 2, 3, 4]

    \end{Verbatim}

    \begin{Verbatim}[commandchars=\\\{\}]
{\color{incolor}In [{\color{incolor}64}]:} \PY{k}{print} \PY{n+nb}{range}\PY{p}{(}\PY{l+m+mi}{4}\PY{p}{)} \PY{c+c1}{\PYZsh{}generate a list}
\end{Verbatim}


    \begin{Verbatim}[commandchars=\\\{\}]
[0, 1, 2, 3]

    \end{Verbatim}

    \begin{Verbatim}[commandchars=\\\{\}]
{\color{incolor}In [{\color{incolor}65}]:} \PY{n}{x} \PY{o}{=} \PY{p}{[}\PY{p}{[}\PY{l+m+mi}{1}\PY{p}{,}\PY{l+m+mi}{2}\PY{p}{]}\PY{p}{,}\PY{p}{[}\PY{l+m+mi}{2}\PY{p}{,}\PY{l+m+mi}{3}\PY{p}{]}\PY{p}{]} \PY{c+c1}{\PYZsh{}multi\PYZhy{}dimensional list}
         \PY{k}{print} \PY{n}{x}\PY{p}{[}\PY{l+m+mi}{0}\PY{p}{]}\PY{p}{[}\PY{l+m+mi}{1}\PY{p}{]}
\end{Verbatim}


    \begin{Verbatim}[commandchars=\\\{\}]
2

    \end{Verbatim}

    \subsubsection{Dictionaries}\label{dictionaries}

    \begin{Verbatim}[commandchars=\\\{\}]
{\color{incolor}In [{\color{incolor}66}]:} \PY{n}{d} \PY{o}{=} \PY{p}{\PYZob{}}\PY{l+s+s1}{\PYZsq{}}\PY{l+s+s1}{a}\PY{l+s+s1}{\PYZsq{}}\PY{p}{:}\PY{l+m+mi}{1}\PY{p}{,}\PY{l+s+s1}{\PYZsq{}}\PY{l+s+s1}{b}\PY{l+s+s1}{\PYZsq{}}\PY{p}{:}\PY{l+m+mi}{2}\PY{p}{\PYZcb{}}
\end{Verbatim}


    \begin{Verbatim}[commandchars=\\\{\}]
{\color{incolor}In [{\color{incolor}67}]:} \PY{k}{print} \PY{n}{d}
\end{Verbatim}


    \begin{Verbatim}[commandchars=\\\{\}]
\{'a': 1, 'b': 2\}

    \end{Verbatim}

    \begin{Verbatim}[commandchars=\\\{\}]
{\color{incolor}In [{\color{incolor}68}]:} \PY{k}{print} \PY{n}{d}\PY{p}{[}\PY{l+s+s1}{\PYZsq{}}\PY{l+s+s1}{a}\PY{l+s+s1}{\PYZsq{}}\PY{p}{]}
\end{Verbatim}


    \begin{Verbatim}[commandchars=\\\{\}]
1

    \end{Verbatim}

    \begin{Verbatim}[commandchars=\\\{\}]
{\color{incolor}In [{\color{incolor}69}]:} \PY{k}{print} \PY{n}{d}\PY{o}{.}\PY{n}{keys}\PY{p}{(}\PY{p}{)}
\end{Verbatim}


    \begin{Verbatim}[commandchars=\\\{\}]
['a', 'b']

    \end{Verbatim}

    \begin{Verbatim}[commandchars=\\\{\}]
{\color{incolor}In [{\color{incolor}70}]:} \PY{k}{print} \PY{n}{d}\PY{o}{.}\PY{n}{values}\PY{p}{(}\PY{p}{)}
\end{Verbatim}


    \begin{Verbatim}[commandchars=\\\{\}]
[1, 2]

    \end{Verbatim}

    Keys can be ints, floats, strings. Values can be these or lists, other
dictionaries, etc.

    \subsection{Math Operators}\label{math-operators}

    \begin{Verbatim}[commandchars=\\\{\}]
{\color{incolor}In [{\color{incolor}71}]:} \PY{n}{x} \PY{o}{=} \PY{l+m+mi}{2}
\end{Verbatim}


    \begin{Verbatim}[commandchars=\\\{\}]
{\color{incolor}In [{\color{incolor}72}]:} \PY{k}{print} \PY{n}{x}\PY{o}{+}\PY{n}{x}\PY{p}{,}\PY{n}{x}\PY{o}{\PYZhy{}}\PY{n}{x}\PY{p}{,}\PY{n}{x}\PY{o}{/}\PY{n}{x}\PY{p}{,}\PY{n}{x}\PY{o}{*}\PY{n}{x}
\end{Verbatim}


    \begin{Verbatim}[commandchars=\\\{\}]
4 0 1 4

    \end{Verbatim}

    \begin{Verbatim}[commandchars=\\\{\}]
{\color{incolor}In [{\color{incolor}73}]:} \PY{k}{print} \PY{n}{x}\PY{o}{*}\PY{o}{*}\PY{l+m+mi}{2} \PY{c+c1}{\PYZsh{} exponent}
\end{Verbatim}


    \begin{Verbatim}[commandchars=\\\{\}]
4

    \end{Verbatim}

    \subsubsection{Working with integer and float
variables}\label{working-with-integer-and-float-variables}

    \begin{Verbatim}[commandchars=\\\{\}]
{\color{incolor}In [{\color{incolor}74}]:} \PY{n}{x}\PY{o}{=}\PY{l+m+mi}{1} \PY{c+c1}{\PYZsh{}integer}
         \PY{n}{y}\PY{o}{=}\PY{l+m+mf}{2.0} \PY{c+c1}{\PYZsh{}float}
         
         \PY{n}{z} \PY{o}{=} \PY{n+nb}{float}\PY{p}{(}\PY{n}{x}\PY{p}{)}
         \PY{n}{w} \PY{o}{=} \PY{n+nb}{int}\PY{p}{(}\PY{n}{y}\PY{p}{)}
\end{Verbatim}


    \begin{Verbatim}[commandchars=\\\{\}]
{\color{incolor}In [{\color{incolor}75}]:} \PY{k}{print} \PY{n}{x}\PY{p}{,}\PY{n}{y}\PY{p}{,}\PY{n}{w}\PY{p}{,}\PY{n}{z}
\end{Verbatim}


    \begin{Verbatim}[commandchars=\\\{\}]
1 2.0 2 1.0

    \end{Verbatim}

    \begin{Verbatim}[commandchars=\\\{\}]
{\color{incolor}In [{\color{incolor}76}]:} \PY{k}{print} \PY{l+s+s1}{\PYZsq{}}\PY{l+s+s1}{1+2.0 =}\PY{l+s+s1}{\PYZsq{}}\PY{p}{,}\PY{n}{x}\PY{o}{+}\PY{n}{y}
         \PY{k}{print} \PY{l+s+s1}{\PYZsq{}}\PY{l+s+s1}{1/2 =}\PY{l+s+s1}{\PYZsq{}}\PY{p}{,}\PY{n}{x}\PY{o}{/}\PY{n}{w}
         \PY{k}{print} \PY{l+s+s1}{\PYZsq{}}\PY{l+s+s1}{1.0/2.0 =}\PY{l+s+s1}{\PYZsq{}}\PY{p}{,}\PY{n}{z}\PY{o}{/}\PY{n}{y}
\end{Verbatim}


    \begin{Verbatim}[commandchars=\\\{\}]
1+2.0 = 3.0
1/2 = 0
1.0/2.0 = 0.5

    \end{Verbatim}

    \subsection{Variable Values and
References}\label{variable-values-and-references}

    Number and string values are copied

    \begin{Verbatim}[commandchars=\\\{\}]
{\color{incolor}In [{\color{incolor}77}]:} \PY{n}{x} \PY{o}{=} \PY{l+m+mf}{4.0}
         \PY{n}{y} \PY{o}{=} \PY{n}{x}
         \PY{n}{x} \PY{o}{=} \PY{l+m+mi}{5}
\end{Verbatim}


    \begin{Verbatim}[commandchars=\\\{\}]
{\color{incolor}In [{\color{incolor}78}]:} \PY{k}{print} \PY{n}{x}\PY{p}{,}\PY{n}{y}
\end{Verbatim}


    \begin{Verbatim}[commandchars=\\\{\}]
5 4.0

    \end{Verbatim}

    Lists, dictionaries and other structures are equated by an address in
memory (by reference) so a change in value in one variable will also
change the value of another variable.

    \begin{Verbatim}[commandchars=\\\{\}]
{\color{incolor}In [{\color{incolor}1}]:} \PY{n}{x} \PY{o}{=} \PY{n+nb}{range}\PY{p}{(}\PY{l+m+mi}{2}\PY{p}{)}
        \PY{n}{y} \PY{o}{=} \PY{n}{x}
        \PY{n}{x}\PY{p}{[}\PY{l+m+mi}{0}\PY{p}{]} \PY{o}{=} \PY{o}{\PYZhy{}}\PY{l+m+mi}{1}
\end{Verbatim}


    \begin{Verbatim}[commandchars=\\\{\}]
{\color{incolor}In [{\color{incolor}2}]:} \PY{k}{print} \PY{n}{x}\PY{p}{,}\PY{n}{y}
\end{Verbatim}


    \begin{Verbatim}[commandchars=\\\{\}]
[-1, 1] [-1, 1]

    \end{Verbatim}

    \subsection{Loops}\label{loops}

    \subsubsection{\texorpdfstring{Standard \emph{for}
Loop}{Standard for Loop}}\label{standard-for-loop}

    \begin{Verbatim}[commandchars=\\\{\}]
{\color{incolor}In [{\color{incolor}81}]:} \PY{k}{for} \PY{n}{i} \PY{o+ow}{in} \PY{n+nb}{range} \PY{p}{(}\PY{l+m+mi}{5}\PY{p}{)}\PY{p}{:}
             \PY{k}{print} \PY{n}{i}\PY{o}{*}\PY{o}{*}\PY{n}{i} \PY{c+c1}{\PYZsh{} indent lines within loops by a tab}
\end{Verbatim}


    \begin{Verbatim}[commandchars=\\\{\}]
1
1
4
27
256

    \end{Verbatim}

    \subsubsection{\texorpdfstring{\emph{for} Loop using
\emph{enumerate}}{for Loop using enumerate}}\label{for-loop-using-enumerate}

    \begin{Verbatim}[commandchars=\\\{\}]
{\color{incolor}In [{\color{incolor}82}]:} \PY{k}{for} \PY{n}{i}\PY{p}{,}\PY{n}{v} \PY{o+ow}{in} \PY{n+nb}{enumerate}\PY{p}{(}\PY{n+nb}{range}\PY{p}{(}\PY{l+m+mi}{5}\PY{p}{)}\PY{p}{)}\PY{p}{:}
             \PY{k}{print} \PY{n}{i}\PY{p}{,}\PY{n}{v}\PY{o}{*}\PY{l+m+mi}{2} \PY{c+c1}{\PYZsh{} prints the index and the value at that index}
\end{Verbatim}


    \begin{Verbatim}[commandchars=\\\{\}]
0 0
1 2
2 4
3 6
4 8

    \end{Verbatim}

    \subsubsection{\texorpdfstring{\emph{for} Loop using
Dictionaries}{for Loop using Dictionaries}}\label{for-loop-using-dictionaries}

    \begin{Verbatim}[commandchars=\\\{\}]
{\color{incolor}In [{\color{incolor}83}]:} \PY{n}{d} \PY{o}{=} \PY{p}{\PYZob{}}\PY{l+s+s1}{\PYZsq{}}\PY{l+s+s1}{a}\PY{l+s+s1}{\PYZsq{}}\PY{p}{:}\PY{l+m+mi}{1}\PY{p}{,}\PY{l+s+s1}{\PYZsq{}}\PY{l+s+s1}{b}\PY{l+s+s1}{\PYZsq{}}\PY{p}{:}\PY{l+m+mi}{2}\PY{p}{\PYZcb{}}
         
         \PY{k}{print} \PY{l+s+s1}{\PYZsq{}}\PY{l+s+s1}{by keys:}\PY{l+s+s1}{\PYZsq{}}
         \PY{k}{for} \PY{n}{key} \PY{o+ow}{in} \PY{n}{d}\PY{o}{.}\PY{n}{keys}\PY{p}{(}\PY{p}{)}\PY{p}{:} \PY{c+c1}{\PYZsh{} or just for key in d}
             \PY{k}{print} \PY{n}{key}\PY{p}{,}\PY{n}{d}\PY{p}{[}\PY{n}{key}\PY{p}{]}
         
             \PY{k}{print} \PY{l+s+s1}{\PYZsq{}}\PY{l+s+s1}{by iteritems:}\PY{l+s+s1}{\PYZsq{}}
         \PY{k}{for} \PY{n}{key}\PY{p}{,}\PY{n}{value} \PY{o+ow}{in} \PY{n}{d}\PY{o}{.}\PY{n}{iteritems}\PY{p}{(}\PY{p}{)}\PY{p}{:}
             \PY{k}{print} \PY{n}{key}\PY{p}{,}\PY{n}{value}
\end{Verbatim}


    \begin{Verbatim}[commandchars=\\\{\}]
by keys:
a 1
by iteritems:
b 2
by iteritems:
a 1
b 2

    \end{Verbatim}

    \subsubsection{List comprehension}\label{list-comprehension}

    \begin{Verbatim}[commandchars=\\\{\}]
{\color{incolor}In [{\color{incolor}84}]:} \PY{n}{result} \PY{o}{=} \PY{p}{[}\PY{n}{i}\PY{o}{+}\PY{n}{i} \PY{k}{for} \PY{n}{i} \PY{o+ow}{in} \PY{n+nb}{range}\PY{p}{(}\PY{l+m+mi}{5}\PY{p}{)}\PY{p}{]}
         \PY{k}{print} \PY{n}{result}
\end{Verbatim}


    \begin{Verbatim}[commandchars=\\\{\}]
[0, 2, 4, 6, 8]

    \end{Verbatim}

    \subsection{Conditionals}\label{conditionals}

    \begin{Verbatim}[commandchars=\\\{\}]
{\color{incolor}In [{\color{incolor}85}]:} \PY{n}{flag} \PY{o}{=} \PY{n+nb+bp}{True}
         \PY{n}{x} \PY{o}{=} \PY{l+m+mi}{1}
         \PY{n}{y} \PY{o}{=} \PY{l+m+mi}{0}
         \PY{k}{if} \PY{n}{flag} \PY{o}{==} \PY{n+nb+bp}{True}\PY{p}{:}
             \PY{k}{print} \PY{l+s+s1}{\PYZsq{}}\PY{l+s+s1}{Hello World}\PY{l+s+s1}{\PYZsq{}}\PY{c+c1}{\PYZsh{} indent lines within ifs by a tab}
             \PY{k}{if} \PY{n}{x} \PY{o}{==} \PY{l+m+mi}{1} \PY{o+ow}{and} \PY{n}{y} \PY{o}{==} \PY{l+m+mi}{1}\PY{p}{:} \PY{c+c1}{\PYZsh{} a nested IF with an AND}
                 \PY{k}{print} \PY{l+s+s1}{\PYZsq{}}\PY{l+s+s1}{AND}\PY{l+s+s1}{\PYZsq{}}
             \PY{k}{if} \PY{n}{x} \PY{o}{==} \PY{l+m+mi}{1} \PY{o+ow}{or} \PY{n}{y} \PY{o}{==} \PY{l+m+mi}{1}\PY{p}{:} \PY{c+c1}{\PYZsh{} a nested IF with an OR}
                 \PY{k}{print} \PY{l+s+s1}{\PYZsq{}}\PY{l+s+s1}{OR}\PY{l+s+s1}{\PYZsq{}}     
\end{Verbatim}


    \begin{Verbatim}[commandchars=\\\{\}]
Hello World
OR

    \end{Verbatim}

    \subsection{Functions}\label{functions}

    \subsubsection{Standard Functions}\label{standard-functions}

    \begin{Verbatim}[commandchars=\\\{\}]
{\color{incolor}In [{\color{incolor}86}]:} \PY{k}{def} \PY{n+nf}{a\PYZus{}function}\PY{p}{(}\PY{n}{x}\PY{p}{)}\PY{p}{:}
             \PY{n}{y} \PY{o}{=} \PY{n}{x}\PY{o}{+}\PY{n}{x} \PY{c+c1}{\PYZsh{} indent lines within functions by a tab}
             \PY{k}{print} \PY{n}{y}
         
         \PY{n}{a\PYZus{}function}\PY{p}{(}\PY{l+m+mi}{5}\PY{p}{)}
         \PY{n}{a\PYZus{}function}\PY{p}{(}\PY{l+s+s1}{\PYZsq{}}\PY{l+s+s1}{Hello World}\PY{l+s+s1}{\PYZsq{}}\PY{p}{)}
\end{Verbatim}


    \begin{Verbatim}[commandchars=\\\{\}]
10
Hello WorldHello World

    \end{Verbatim}

    \subsubsection{Functions that return
values}\label{functions-that-return-values}

    \begin{Verbatim}[commandchars=\\\{\}]
{\color{incolor}In [{\color{incolor}87}]:} \PY{k}{def} \PY{n+nf}{b\PYZus{}function}\PY{p}{(}\PY{n}{x}\PY{p}{)}\PY{p}{:}
             \PY{k}{return} \PY{n}{x}\PY{o}{+}\PY{n}{x}
         
         \PY{k}{print} \PY{n}{b\PYZus{}function}\PY{p}{(}\PY{l+m+mi}{5}\PY{p}{)}
\end{Verbatim}


    \begin{Verbatim}[commandchars=\\\{\}]
10

    \end{Verbatim}

    \subsubsection{\texorpdfstring{Shorthand Functions
(\emph{lambda})}{Shorthand Functions (lambda)}}\label{shorthand-functions-lambda}

    \begin{Verbatim}[commandchars=\\\{\}]
{\color{incolor}In [{\color{incolor}88}]:} \PY{n}{a\PYZus{}function2} \PY{o}{=} \PY{k}{lambda} \PY{n}{x}\PY{p}{:} \PY{n}{x}\PY{o}{+}\PY{n}{x}
         
         \PY{k}{print} \PY{n}{a\PYZus{}function2}\PY{p}{(}\PY{l+m+mi}{5}\PY{p}{)}
         \PY{k}{print} \PY{n}{a\PYZus{}function2}\PY{p}{(}\PY{l+s+s1}{\PYZsq{}}\PY{l+s+s1}{Hello, World}\PY{l+s+s1}{\PYZsq{}}\PY{p}{)}
\end{Verbatim}


    \begin{Verbatim}[commandchars=\\\{\}]
10
Hello, WorldHello, World

    \end{Verbatim}

    \subsection{Imports}\label{imports}

Typically library imports are made at the top of your notebook

    \begin{Verbatim}[commandchars=\\\{\}]
{\color{incolor}In [{\color{incolor}89}]:} \PY{c+c1}{\PYZsh{} Importing}
         
         \PY{c+c1}{\PYZsh{} all functions and classes in numpy are accessed using np.\PYZpc{}name\PYZpc{}}
         \PY{k+kn}{import} \PY{n+nn}{numpy} \PY{k+kn}{as} \PY{n+nn}{np}
         
         \PY{n}{x} \PY{o}{=} \PY{n}{np}\PY{o}{.}\PY{n}{array}\PY{p}{(}\PY{p}{[}\PY{l+m+mi}{1}\PY{p}{,}\PY{l+m+mi}{2}\PY{p}{,}\PY{l+m+mi}{3}\PY{p}{]}\PY{p}{)}
         \PY{n}{y} \PY{o}{=} \PY{n}{x} \PY{c+c1}{\PYZsh{} sets the reference}
         
         \PY{c+c1}{\PYZsh{} importing the function copy in the library copy }
         \PY{k+kn}{from} \PY{n+nn}{copy} \PY{k+kn}{import} \PY{n}{copy}
         
         \PY{n}{z} \PY{o}{=} \PY{n}{copy}\PY{p}{(}\PY{n}{x}\PY{p}{)}
         \PY{n}{x}\PY{p}{[}\PY{l+m+mi}{0}\PY{p}{]} \PY{o}{=} \PY{o}{\PYZhy{}}\PY{l+m+mi}{1}
         
         \PY{k}{print} \PY{n}{x}
         \PY{k}{print} \PY{n}{y}
         \PY{k}{print} \PY{n}{z}
\end{Verbatim}


    \begin{Verbatim}[commandchars=\\\{\}]
[-1  2  3]
[-1  2  3]
[1 2 3]

    \end{Verbatim}

    \begin{Verbatim}[commandchars=\\\{\}]
{\color{incolor}In [{\color{incolor}6}]:} \PY{c+c1}{\PYZsh{} Usefl imports}
        \PY{k+kn}{import} \PY{n+nn}{numpy} \PY{k+kn}{as} \PY{n+nn}{np} \PY{c+c1}{\PYZsh{} for array/matrix manipulation}
        \PY{k+kn}{import} \PY{n+nn}{scipy} \PY{k+kn}{as} \PY{n+nn}{sp} \PY{c+c1}{\PYZsh{} numerical routines}
        \PY{k+kn}{import} \PY{n+nn}{matplotlib.pyplot} \PY{k+kn}{as} \PY{n+nn}{plt} \PY{c+c1}{\PYZsh{} for creating graphs}
        \PY{k+kn}{import} \PY{n+nn}{seaborn} \PY{k+kn}{as} \PY{n+nn}{sns} \PY{c+c1}{\PYZsh{} for creating nice graphs quickly}
        \PY{k+kn}{import} \PY{n+nn}{pandas} \PY{k+kn}{as} \PY{n+nn}{pd} \PY{c+c1}{\PYZsh{} for reading/writing tabular data}
\end{Verbatim}


    \subsection{Helpful configurations}\label{helpful-configurations}

    \begin{Verbatim}[commandchars=\\\{\}]
{\color{incolor}In [{\color{incolor}4}]:} \PY{n}{np}\PY{o}{.}\PY{n}{set\PYZus{}printoptions}\PY{p}{(}\PY{n}{precision}\PY{o}{=}\PY{l+m+mi}{3}\PY{p}{)} \PY{c+c1}{\PYZsh{} print 3 sig. digits in arrays}
        
        \PY{c+c1}{\PYZsh{} So graphs are presented inline when a cell is run:}
        \PY{o}{\PYZpc{}}\PY{k}{matplotlib} inline 
        
        \PY{n}{mpl}\PY{o}{.}\PY{n}{rcParams}\PY{p}{[}\PY{l+s+s1}{\PYZsq{}}\PY{l+s+s1}{figure.figsize}\PY{l+s+s1}{\PYZsq{}}\PY{p}{]} \PY{o}{=} \PY{p}{[}\PY{l+m+mf}{10.4}\PY{p}{,} \PY{l+m+mf}{7.15}\PY{p}{]} \PY{c+c1}{\PYZsh{} Set figure/graph size}
        \PY{n}{mpl}\PY{o}{.}\PY{n}{rcParams}\PY{p}{[}\PY{l+s+s1}{\PYZsq{}}\PY{l+s+s1}{font.size}\PY{l+s+s1}{\PYZsq{}}\PY{p}{]} \PY{o}{=} \PY{l+m+mi}{20} \PY{c+c1}{\PYZsh{} Set figure/graph font size}
        \PY{n}{mpl}\PY{o}{.}\PY{n}{rcParams}\PY{p}{[}\PY{l+s+s1}{\PYZsq{}}\PY{l+s+s1}{text.usetex}\PY{l+s+s1}{\PYZsq{}}\PY{p}{]} \PY{o}{=} \PY{n+nb+bp}{True} \PY{c+c1}{\PYZsh{} Optional (using Latex in graphs)}
        
        \PY{n}{sns}\PY{o}{.}\PY{n}{set\PYZus{}style}\PY{p}{(}\PY{l+s+s1}{\PYZsq{}}\PY{l+s+s1}{white}\PY{l+s+s1}{\PYZsq{}}\PY{p}{)} \PY{c+c1}{\PYZsh{} Recommended configuration for seaborn}
        \PY{n}{sns}\PY{o}{.}\PY{n}{set\PYZus{}context}\PY{p}{(}\PY{l+s+s1}{\PYZsq{}}\PY{l+s+s1}{talk}\PY{l+s+s1}{\PYZsq{}}\PY{p}{)} \PY{c+c1}{\PYZsh{} Recommended configuration for seaborn}
        
        \PY{c+c1}{\PYZsh{} insert a random number here to seed the (psuedo) random number generator}
        \PY{n}{np}\PY{o}{.}\PY{n}{random}\PY{o}{.}\PY{n}{seed}\PY{p}{(}\PY{l+m+mi}{847}\PY{p}{)} \PY{c+c1}{\PYZsh{} put in your own random number}
\end{Verbatim}


    \subsection{Intro to Numpy}\label{intro-to-numpy}

    \subsubsection{Numpy Math}\label{numpy-math}

    \begin{Verbatim}[commandchars=\\\{\}]
{\color{incolor}In [{\color{incolor}92}]:} \PY{k}{print} \PY{n}{np}\PY{o}{.}\PY{n}{pi}
\end{Verbatim}


    \begin{Verbatim}[commandchars=\\\{\}]
3.14159265359

    \end{Verbatim}

    \begin{Verbatim}[commandchars=\\\{\}]
{\color{incolor}In [{\color{incolor}93}]:} \PY{k}{print} \PY{n}{np}\PY{o}{.}\PY{n}{exp}\PY{p}{(}\PY{l+m+mi}{1}\PY{p}{)}
\end{Verbatim}


    \begin{Verbatim}[commandchars=\\\{\}]
2.71828182846

    \end{Verbatim}

    \begin{Verbatim}[commandchars=\\\{\}]
{\color{incolor}In [{\color{incolor}94}]:} \PY{k}{print} \PY{n}{np}\PY{o}{.}\PY{n}{log}\PY{p}{(}\PY{n}{np}\PY{o}{.}\PY{n}{exp}\PY{p}{(}\PY{l+m+mi}{1}\PY{p}{)}\PY{p}{)} \PY{c+c1}{\PYZsh{}natural logarithm}
\end{Verbatim}


    \begin{Verbatim}[commandchars=\\\{\}]
1.0

    \end{Verbatim}

    \begin{Verbatim}[commandchars=\\\{\}]
{\color{incolor}In [{\color{incolor}95}]:} \PY{k}{print} \PY{n}{np}\PY{o}{.}\PY{n}{sqrt}\PY{p}{(}\PY{l+m+mi}{2}\PY{p}{)}
\end{Verbatim}


    \begin{Verbatim}[commandchars=\\\{\}]
1.41421356237

    \end{Verbatim}

    \subsubsection{Numpy Arrays and
Matrices}\label{numpy-arrays-and-matrices}

Like lists, but better for numerical calculations

    \begin{Verbatim}[commandchars=\\\{\}]
{\color{incolor}In [{\color{incolor}96}]:} \PY{k}{print} \PY{n}{np}\PY{o}{.}\PY{n}{array}\PY{p}{(}\PY{p}{[}\PY{l+m+mi}{1}\PY{p}{,}\PY{l+m+mi}{2}\PY{p}{,}\PY{l+m+mi}{3}\PY{p}{]}\PY{p}{)}
         \PY{k}{print} \PY{n}{np}\PY{o}{.}\PY{n}{matrix}\PY{p}{(}\PY{p}{[}\PY{p}{[}\PY{l+m+mi}{1}\PY{p}{,}\PY{l+m+mi}{2}\PY{p}{]}\PY{p}{,}\PY{p}{[}\PY{l+m+mi}{3}\PY{p}{,}\PY{l+m+mi}{4}\PY{p}{]}\PY{p}{]}\PY{p}{)} \PY{c+c1}{\PYZsh{} we won\PYZsq{}t use these mutch}
\end{Verbatim}


    \begin{Verbatim}[commandchars=\\\{\}]
[1 2 3]
[[1 2]
 [3 4]]

    \end{Verbatim}

    \begin{Verbatim}[commandchars=\\\{\}]
{\color{incolor}In [{\color{incolor}97}]:} \PY{k}{print} \PY{n}{np}\PY{o}{.}\PY{n}{arange}\PY{p}{(}\PY{l+m+mi}{3}\PY{p}{)}
\end{Verbatim}


    \begin{Verbatim}[commandchars=\\\{\}]
[0 1 2]

    \end{Verbatim}

    \begin{Verbatim}[commandchars=\\\{\}]
{\color{incolor}In [{\color{incolor}98}]:} \PY{k}{print} \PY{n}{np}\PY{o}{.}\PY{n}{ones}\PY{p}{(}\PY{l+m+mi}{3}\PY{p}{)}\PY{p}{,} \PY{n}{np}\PY{o}{.}\PY{n}{zeros}\PY{p}{(}\PY{l+m+mi}{3}\PY{p}{)}
\end{Verbatim}


    \begin{Verbatim}[commandchars=\\\{\}]
[ 1.  1.  1.] [ 0.  0.  0.]

    \end{Verbatim}

    \begin{Verbatim}[commandchars=\\\{\}]
{\color{incolor}In [{\color{incolor}99}]:} \PY{k}{print} \PY{n}{np}\PY{o}{.}\PY{n}{linspace}\PY{p}{(}\PY{l+m+mi}{4}\PY{p}{,}\PY{l+m+mi}{8}\PY{p}{,}\PY{l+m+mi}{3}\PY{p}{)}
\end{Verbatim}


    \begin{Verbatim}[commandchars=\\\{\}]
[ 4.  6.  8.]

    \end{Verbatim}

    Experiment with other array functions: concatenate, +, - add, subtract,
multiply, divide

    \subsection{Random Numbers and
Plotting}\label{random-numbers-and-plotting}

    \begin{Verbatim}[commandchars=\\\{\}]
{\color{incolor}In [{\color{incolor}7}]:} \PY{c+c1}{\PYZsh{} random numbers betweeen 0 and 1, other distributions available in np.random}
        \PY{n}{x} \PY{o}{=} \PY{n}{np}\PY{o}{.}\PY{n}{random}\PY{o}{.}\PY{n}{random}\PY{p}{(}\PY{l+m+mi}{1000}\PY{p}{)}
        \PY{n}{y} \PY{o}{=} \PY{n}{np}\PY{o}{.}\PY{n}{random}\PY{o}{.}\PY{n}{random}\PY{p}{(}\PY{l+m+mi}{1000}\PY{p}{)}
        
        \PY{n}{fig}\PY{p}{,} \PY{n}{axes} \PY{o}{=} \PY{n}{plt}\PY{o}{.}\PY{n}{subplots}\PY{p}{(}\PY{n}{nrows}\PY{o}{=}\PY{l+m+mi}{1}\PY{p}{,} \PY{n}{ncols}\PY{o}{=}\PY{l+m+mi}{2}\PY{p}{,} \PY{n}{figsize}\PY{o}{=}\PY{p}{(}\PY{l+m+mf}{10.4}\PY{p}{,}\PY{l+m+mf}{3.5}\PY{p}{)}\PY{p}{)}
        
        \PY{n}{axes}\PY{p}{[}\PY{l+m+mi}{0}\PY{p}{]}\PY{o}{.}\PY{n}{plot}\PY{p}{(}\PY{n}{x}\PY{p}{,}\PY{n}{y}\PY{p}{,}\PY{l+s+s1}{\PYZsq{}}\PY{l+s+s1}{.}\PY{l+s+s1}{\PYZsq{}}\PY{p}{)}\PY{p}{;} \PY{c+c1}{\PYZsh{} the semi\PYZhy{}colon hides an uneeded output}
        \PY{n}{axes}\PY{p}{[}\PY{l+m+mi}{0}\PY{p}{]}\PY{o}{.}\PY{n}{set}\PY{p}{(}\PY{n}{title}\PY{o}{=}\PY{l+s+s1}{\PYZsq{}}\PY{l+s+s1}{Uniform distribution samples}\PY{l+s+s1}{\PYZsq{}}\PY{p}{,} \PY{n}{xlabel}\PY{o}{=}\PY{l+s+s1}{\PYZsq{}}\PY{l+s+s1}{X}\PY{l+s+s1}{\PYZsq{}}\PY{p}{,} \PY{n}{ylabel}\PY{o}{=}\PY{l+s+s1}{\PYZsq{}}\PY{l+s+s1}{Y}\PY{l+s+s1}{\PYZsq{}}\PY{p}{)}\PY{p}{;}
        
        \PY{n}{sns}\PY{o}{.}\PY{n}{distplot}\PY{p}{(}\PY{n}{x}\PY{p}{,} \PY{n}{kde}\PY{o}{=}\PY{n+nb+bp}{False}\PY{p}{,} \PY{n}{ax}\PY{o}{=}\PY{n}{axes}\PY{p}{[}\PY{l+m+mi}{1}\PY{p}{]}\PY{p}{)}\PY{p}{;} \PY{c+c1}{\PYZsh{} histogram}
        \PY{n}{axes}\PY{p}{[}\PY{l+m+mi}{1}\PY{p}{]}\PY{o}{.}\PY{n}{set}\PY{p}{(}\PY{n}{title}\PY{o}{=}\PY{l+s+s1}{\PYZsq{}}\PY{l+s+s1}{Histogram of Samples}\PY{l+s+s1}{\PYZsq{}}\PY{p}{,} \PY{n}{xlabel}\PY{o}{=}\PY{l+s+s1}{\PYZsq{}}\PY{l+s+s1}{x}\PY{l+s+s1}{\PYZsq{}}\PY{p}{,} 
                    \PY{n}{ylabel}\PY{o}{=}\PY{l+s+s1}{\PYZsq{}}\PY{l+s+s1}{Occurrences}\PY{l+s+s1}{\PYZsq{}}\PY{p}{)}\PY{p}{;}
\end{Verbatim}


    \begin{center}
    \adjustimage{max size={0.9\linewidth}{0.9\paperheight}}{output_96_0.png}
    \end{center}
    { \hspace*{\fill} \\}
    
    \subsection{Pandas}\label{pandas}

Read, write and work with tabular files

    \begin{Verbatim}[commandchars=\\\{\}]
{\color{incolor}In [{\color{incolor}9}]:} \PY{n}{df} \PY{o}{=} \PY{n}{pd}\PY{o}{.}\PY{n}{read\PYZus{}csv}\PY{p}{(}\PY{l+s+s1}{\PYZsq{}}\PY{l+s+s1}{../data/randomdata.csv}\PY{l+s+s1}{\PYZsq{}}\PY{p}{)}
        \PY{k}{print} \PY{n}{df}\PY{p}{[}\PY{l+m+mi}{0}\PY{p}{:}\PY{l+m+mi}{10}\PY{p}{]}
\end{Verbatim}


    \begin{Verbatim}[commandchars=\\\{\}]
          x         y
0  0.102258  0.113201
1  0.474799  0.595240
2  0.414108  0.572583
3  0.590336  0.450980
4  0.194600  0.330423
5  0.180427  0.556806
6  0.785828  0.499291
7  0.260421  0.498740
8  0.540955  0.623766
9  0.998440  0.617484

    \end{Verbatim}

    \begin{Verbatim}[commandchars=\\\{\}]
{\color{incolor}In [{\color{incolor}102}]:} \PY{k}{print} \PY{n+nb}{type}\PY{p}{(}\PY{n}{df}\PY{p}{[}\PY{l+s+s1}{\PYZsq{}}\PY{l+s+s1}{x}\PY{l+s+s1}{\PYZsq{}}\PY{p}{]}\PY{o}{.}\PY{n}{values}\PY{p}{)}
\end{Verbatim}


    \begin{Verbatim}[commandchars=\\\{\}]
<type 'numpy.ndarray'>

    \end{Verbatim}


    % Add a bibliography block to the postdoc
    
    
    
    \end{document}
